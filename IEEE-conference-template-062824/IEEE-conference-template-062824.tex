\documentclass[conference]{IEEEtran}
\IEEEoverridecommandlockouts
% The preceding line is only needed to identify funding in the first footnote. If that is unneeded, please comment it out.
%Template version as of 6/27/2024

\usepackage{cite}
\usepackage{amsmath,amssymb,amsfonts}
\usepackage{algorithmic}
\usepackage{graphicx}
\usepackage{textcomp}
\usepackage{xcolor}
\def\BibTeX{{\rm B\kern-.05em{\sc i\kern-.025em b}\kern-.08em
    T\kern-.1667em\lower.7ex\hbox{E}\kern-.125emX}}
\begin{document}

\title{Game Obsession and Player engagement focused on mobile gaming enviroments*\\
{\footnotesize \textsuperscript{*}Note: Sub-titles are not captured for https://ieeexplore.ieee.org  and
should not be used}
%\thanks{Identify applicable funding agency here. If none, delete this.}
}

\author{\IEEEauthorblockN{1\textsuperscript{st} Marshall Sharp}
\IEEEauthorblockA{\textit{Games Academy} \\
\textit{Falmouth University}\\
Falmouth, United Kindgom \\
MS279226@falmouth.ac.uk}
}

\maketitle



\begin{abstract}

\end{abstract}

\begin{IEEEkeywords}
engagement, obsession, mobile games
\end{IEEEkeywords}

\section{Introduction}
The nature of obsession, most notably addiction, is a difficult thing to define concisely, as we don't know for certain what classifies as an addiction in an age of technological advancements. Over the years, the definition for addiction has been made to accomodate something which, in theory, cannot be applicable to it's original definition. It is most prevelent in the games industry, with game addiction being classified as a medical condition. This, however, could be flawed, this paper will take a look at game obsession instead of addiction, and go over game addiciton and player enagement based on game obsession instead of addiction, and challenging the classification of such terms. 
\section{Background}
\subsection{Defining Obsession and why it's different from addiction}
Gaming addiction can be defined in differing ways. One paper \cite{yasir2021} defines gaming addiction with the uses and gratification theory and defines overall addiction with the medical paradigm, which has a larger emphasis on biological and psychological dependacies, which has been retrofitted to accompany the advancements of technology, and specifically in the mobile games industry. this paper also mentions that another study conducted came to the conclusion that it can be defined as the regular action of taking drastic actions in the game, such as buying in game goods or features in a freenium game enviroment \cite{XWang2021}.

Gaming Obsession is differing to Gaming addiction because.
\subsection{Impacts of Game Obsession}
\subsection{Causes of Game Obsession}

\section{Methodology}
\subsection{Ethics and safety}
\subsection {Hypothesis}
\subsection{Questions}
\subsection {Stats testing}


\section{Results}

\section {Conclusion}

\section*{References}

\bibliographystyle{plain}
\bibliography{Dissertationbib.bib}


\section {Addendum}

\end{document}
