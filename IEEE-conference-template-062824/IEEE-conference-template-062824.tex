\documentclass[conference]{IEEEtran}
\IEEEoverridecommandlockouts
% The preceding line is only needed to identify funding in the first footnote. If that is unneeded, please comment it out.
%Template version as of 6/27/2024

\usepackage{cite}
\usepackage{amsmath,amssymb,amsfonts}
%\usepackage{algcompatible}
\usepackage{algorithm}
\usepackage{algorithmic}
%\usepackage{algpseudocode}
\usepackage{hyperref}
\usepackage{graphicx}
\usepackage{textcomp}
\usepackage{float}
\usepackage{xcolor}
\graphicspath{{./images/}}
\def\BibTeX{{\rm B\kern-.05em{\sc i\kern-.025em b}\kern-.08em
    T\kern-.1667em\lower.7ex\hbox{E}\kern-.125emX}}
\begin{document}

\title{Game Obsession and Player engagement focused on mobile gaming enviroments\\
%{\footnotesize \textsuperscript{*}Note: Sub-titles are not captured for https://ieeexplore.ieee.org  and
%should not be used}
%\thanks{Identify applicable funding agency here. If none, delete this.}
}

\author{\IEEEauthorblockN{1\textsuperscript{st} Marshall Sharp}
\IEEEauthorblockA{\textit{Games Academy} \\
\textit{Falmouth University}\\
Falmouth, United Kindgom \\
MS279226@falmouth.ac.uk}
}

\maketitle



\begin{abstract}

\end{abstract}

\begin{IEEEkeywords}
engagement, obsession, mobile games
\end{IEEEkeywords}

\section{Introduction}
The nature of obsession, most notably addiction, is a difficult thing to define concisely, as we don't know for certain what classifies as an addiction in an age of technological advancements. Over the years, the definition for addiction has been made to accomodate something which, in theory, cannot be applicable to it's original definition. It is most prevelent in the games industry, with game addiction being classified as a medical condition. This, however, could be flawed, this paper will take a look at game obsession instead of addiction, and go over game addiciton and player enagement based on game obsession instead of addiction, and challenging the classification of such terms.\\
\section{Background}
Gaming addiction can be defined in differing ways. One paper \cite{yasir2021} defines gaming addiction with the uses and gratification theory and defines overall addiction with the medical paradigm, which has a larger emphasis on biological and psychological dependacies, which has been retrofitted to accompany the advancements of technology, and specifically in the mobile games industry. this paper also mentions that another study conducted came to the conclusion that it can be defined as the regular action of taking drastic actions in the game, such as buying in game goods or features in a freenium game enviroment \cite{XWang2021}. \\

\cite{Naaj2021} was conducted during the midst of the COVID-19 pandemic in early 2020 at Ajman University, where it was found that from a sample size of 317 that have participated in an interview. They were asked three questions: \\

1) "On average how many hours do you spend playing video games daily?"\\
This question had 17\%  of the participants didn't play video games, though out of the participants that have, 49\% play around 2 hours on average, with 6\%  playing 8 hours and 3\% play for 10 hours. The remaining 25\%  for 5 hours as an average daily. Using this information, it can be assumed gaming addicition affects 9\% if we're only going off of the hours one plays daily. It can assumed based on this question alone that what causes game addiction is the amount of time played for the participant.\\

2) "Does gaming overall give you a real fulfilment in life?"\\
51\% of the 317 people interviewed said "yes", 10\% felt indifferent and the remaining 39\% said "no".  This question contradicts question 1 of this study, as the question this poses reevaluates game addiction not based on time played but the engagement one feels for the game. The \% of people who play for 2 hours is strikingly close to how many people said "yes" to this question, so this could also be used to assume that it's about how enjaoyable and engaging the game is, over how much content and play time is in the game.\\

3) "If it were needed, could you quit playing video games easily?"\\
58\% of the participating body said "yes", whilst 39\% said "no". This gives credence to the assumptions made using question 2, as over half the participants can safely withdraw from gaming entirely. However, with the quantity of participants who say they can't, it could also be assumed that it can be considered to be somewhat unreliable, as there's no proper discernable evidence to say definitively from the interview conducted.\\ 
 
4)"Did the average daily time you spend playing video games increase during [the] COVID-19 Pandemic?"\\
almost half those who participated said "no", at 49\%. This also reinforces the assumptions made by question 2, though the volume in the "yes/no" \% is similar to question 3, with those who have said "yes" have a \% of 34. It could also be considered inconclusive.

Overall, the interview of participants states that the Average is between 2 hours and 5 hours, Most people find fulfillment in playing games, and it is just as easy to leave gaming as it is to stay, with there being no impact on people's habits over the COVID-19 Pandemic and subsequent lockdowns.\\

Naaj et al also did ANOVA and T-tests on their hypthosises. All hypothesis revolving around these tests result in the rejection of them, both with a power of 0, the same number of participants, but that is where the similarities end. Both tables in the paper revealed that more men play video games and it does indeed affect academic level's of the students. Table one has a Lower standard deviation and higher mean for men compared to their female counterparts. This is also an similar case in table two of the paper, with females getting equal mean and deviation. This can be seen as there being little to no effect on women when it comes to gaming.\\

ANOVA tests were used to create graphs, which are figure 2 and figure 3 for this paper. both of which depict a negative corraltation in relation to CGPA. It can be assumed that students who participated in this study have a higher CGPA the less time they spend playing video games \cite{Naaj2021}.
% \cite{Rahman2021}. talk about effects on sleep, and compare the time statistics to naaj's paper
% \cite{Jing2024}. talk about the EEG and define that. in addition, talk about the differnces in High Vulnerbility game addicts and Low vulnerbility game addicts.
% \cite{}. talk about
% \cite{}. talk about
% \cite{}. talk about
% \cite{}. talk about
% \cite{}. talk about
% \cite{}. talk about

From a medical standpoint, The NHS in hampshire defines gaming addiction as a variety of symptoms, with the main symptoms being reclusion from the wider world, poor hygine or lack of attention to basic needs (i.e washing, nourishments such as food or water, or sleep) and hyperfixation on gaming \cite{NHSHamp24}. This definion can be seen as disagreeable due to the medical paradigm being more on a biological level rather than technological. In addition, it fails to take into consideration to other conditions in terms of symptoms, such as ADHD, Autism, and other neurodivergancies, Which are defined in the text revision of the 5th edition Diagnostic and Statistical manual of mental disorders \cite{Association2022}.\\

The term "Gaming Obsession" could fit this role better. One paper looks into Game Loyalty and Purchase intention \cite{Ramli2022}, in which it was found that Players who are loyal to the games they play are contradictory to a differing source they cited\cite{Widodo2020}, which has a trait of compulsive buying, which mirrors points made in \cite {yasir2021}.  This study was based on 350 out of 680 people, with the majority of the selected amount (65\%) being men and the remaining 35\% being women. the study found that from both of these groups, 55\% of the total play league for 2 -3 hours everyday. The NHS  \cite{NHS2021} defines obsession as "Unwanted or unpleasent thought, image or urge that repeatedly enters your mind, causing feelings of anxiety, digust or unease," which is in the context of Obessive Compulsive Disorder (OCD), though can be applied to gaming and other external areas.\\

\section{Methodology}
The Experiment for this research will involve creating a game similar to a Idle Game, Similar to games like cookie clicker \footnote{\url {https://store.steampowered.com/app/1454400/Cookie_Clicker/}}, Where the Player can log in every so often to play the game and it requires bare minimum interaction from the player. According to \cite{Hwang2024}. Each participant for each participant  is half an hour. In that time, They will play the game for that time, or until they request that it ends early. An interview will happen after, going over the player's experiance with the game. Time played will be recorded. This Experiment will last three weeks. The Artefact for the experiment will be made in Unity Version 2023.2.20 \footnote{\url {https://unity.com/}} and stored in this repository:\\

\url{https://github.falmouth.ac.uk/MS279226/Dissertation-MS279226.git}\\

Data Will be recorded in a Comma seperated Values file (.CSV). This file format makes it easier to format into graphs using programming languages such as R and Python. If utilising python to create the graphs, Pandas, Numpy and other plugins and add-ons would need to be used.

\subsection{Ethics and safety}
There are some things to account for with this research, to do with people and their safety. The experiment is in the hands of the participant, with them being required to read a information form and sign a consent form. They can request the experiment to end early, If they no longer consent to the experiment.  If they no longer consent, data will not be used when drawing the conclusion and instead will be destroyed. Despite there being little physical risk, a risk assessment will still need to be carried out, with the removal of items that can be harmful for participants taking part in the research, or the addition of items that can mitigate harm/ provide ease of use with the artifact.\\
  
\subsection {Hypothesis}
There are two main hypothesis with this experiment, with a third one to account for the possibility of it happening:

H1: The Player will have a hyperfocus on the game, and/or fulfill most of the criteria of game addiction mentioned in \cite{NHSHamp24}, though after being told it's over or requesting, they do not see a reason to keep going.\\

H2: The Player will not fulfill any criteria, and so there is no change plausible.\\

H2 This is inline with Nursafika's paper \cite{Nursafika2024}, which is a study using stumble guys, a competitive party game where you compete in various minigames whilst being subjected to silly in-game physics. With the use of the Game Experiance Questionnaire (GEQ) method, they found that 61.46\% of participants could return to reality after playing the game, 58.30\% felt immersed in the game and 57\% felt the game had a negative affect on them, which can be see as a lesser effect when comparing that result to the 77.28\% of participants who felt the game has had a positive effect on them. at least half of all other values were above 60\%,  with the conclusion that stumble guys has little to no negative impact or addictive nature to players of stumble guys.\\

H3: The Player will fulfill the criteria of H1, but will continue with the experiment, even after it's over either through the duration being up or from requesting it to stop. This will fulfill all of the criteria in \cite{NHSHamp24}, and be classed as addiciton. \\
\subsection{Questions}
There are three main questions with this research:

Q1: How Long does it take for someone to consider themselves "Obessed" with the game?

Q2: Could/Should "Online Gaming Addiction" be renamed to "Online Gaming Obsession"?

Q3: What truly defines addiction? could addiction and obsession be one in the same?\\
\subsection {Stats testing}
Utilising G*Power, A T-test using the Point Biseral Model was conducted to determine the sample size for the experiment. Figure 1 Shows that the reccomended sample size using a medium effect size (0.3), a 0.05 error probability and 0.95 Power will result in the sample size being 111.  With the reccomended amount being about 100 participants, and most sources have 200 or more, this is on the smaller end. Python will also be used to construct the graphs for time played, and compares the time played with the atrefact to how long on average the participant plays games, if any.\\

The code below is how it will be plotted:
\begin{algorithm}
\caption{The code utilised to show how the data will be plotted in python}
\begin{algorithmic}
\STATE $ graph data \gets .CSV $
\FOR {$each $ $entry$ $in$ $.CSV$}
	\STATE $Plot X-axis$
	\STATE $Plot Y-axis$
	\STATE $Plot Graph Points$
\ENDFOR
\STATE $print graph to screen$
\end{algorithmic}
\end{algorithm}

\begin{figure}[H]
\includegraphics[width =0.5 \textwidth]{fig1(2)}
\caption{The reccomended sample size according to the G*Power t-test}
\end{figure}

A Graph of the G*Power results was also plotted in the program, which portrays the increase in sample size and it's corralation to the Power. this is shown in figure 2. In addition, it can be anticipated that the sample size will more accurately be around 50 people, due to the restriction of keeping the participants to the university rather than the broader public. 

\begin{figure}[H]
\includegraphics[width = 0.5\textwidth]{fig2(2)}
\caption{The reccomended sample size according to the G*Power t-test plotted as a graph, with X as the Power and Y as the sample size}
\end{figure}

\subsection {Restraints}
One restraint with this research is sample size. Due to my sample being students within the university. This limits the findings heavily, with there not being to get the sample size that will give a more noteworthy effect. In addition, the sample size will be limited even more due to the time period for the experiment only being three weeks. This is a similar pitfall to \cite{Naaj2021}.\\


%\section{Results}

%\section {Conclusion}


\bibliographystyle{IEEEtran}
\bibliography{Dissertationbib.bib}

\end{document}
