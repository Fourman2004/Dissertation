\documentclass[conference]{IEEEtran}
\IEEEoverridecommandlockouts
% The preceding line is only needed to identify funding in the first footnote. If that is unneeded, please comment it out.
%Template version as of 6/27/2024

\usepackage{cite}
\usepackage{amsmath,amssymb,amsfonts}
\usepackage{algorithmic}
\usepackage{graphicx}
\usepackage{textcomp}
\usepackage{xcolor}
\def\BibTeX{{\rm B\kern-.05em{\sc i\kern-.025em b}\kern-.08em
    T\kern-.1667em\lower.7ex\hbox{E}\kern-.125emX}}
\begin{document}

\title{Game Obsession and Player engagement focused on mobile gaming enviroments*\\
{\footnotesize \textsuperscript{*}Note: Sub-titles are not captured for https://ieeexplore.ieee.org  and
should not be used}
%\thanks{Identify applicable funding agency here. If none, delete this.}
}

\author{\IEEEauthorblockN{1\textsuperscript{st} Marshall Sharp}
\IEEEauthorblockA{\textit{Games Academy} \\
\textit{Falmouth University}\\
Falmouth, United Kindgom \\
MS279226@falmouth.ac.uk}
}

\maketitle



\begin{abstract}

\end{abstract}

\begin{IEEEkeywords}
engagement, obsession, mobile games
\end{IEEEkeywords}

\section{Introduction}
The nature of obsession, most notably addiction, is a difficult thing to define concisely, as we don't know for certain what classifies as an addiction in an age of technological advancements. Over the years, the definition for addiction has been made to accomodate something which, in theory, cannot be applicable to it's original definition. It is most prevelent in the games industry, with game addiction being classified as a medical condition. This, however, could be flawed, this paper will take a look at game obsession instead of addiction, and go over game addiciton and player enagement based on game obsession instead of addiction, and challenging the classification of such terms. 
\section{Background}
\subsection{Defining Obsession and why it's different from addiction}
Gaming addiction can be defined in differing ways. One paper \cite{yasir2021} defines gaming addiction with the uses and gratification theory and defines overall addiction with the medical paradigm, which has a larger emphasis on biological and psychological dependacies, which has been retrofitted to accompany the advancements of technology, and specifically in the mobile games industry. this paper also mentions that another study conducted came to the conclusion that it can be defined as the regular action of taking drastic actions in the game, such as buying in game goods or features in a freenium game enviroment \cite{XWang2021}.

The NHS in hampshire defines gaming addiction as a variety of symptoms, with the main symptoms being reclusion from the wider world, poor hygine or lack of attention to basic needs (i.e washing, nourishments such as food or water, or sleep) and hyperfixation on gaming \cite{NHSHamp24}. This definion can be seen as disagreeable due to the medical paradigm being more on a biological level rather than technological. In addition, it fails to take into consideration to other conditions in terms of symptoms, such as ADHD, Autism, and other neurodivergancies.

The term "Gaming Obsession" could fit this role better. One paper looks into Game Loyalty and Purchase intention \cite{Ramli2022}, in which it was found that Players who are loyal to the games they play are contradictory to a differing source they cited\cite{Widodo2020}, which has a trait of compulsive buying, which mirrors points made in \cite {yasir2021}.  This study was based on 350 out of 680 people, with the majority of the selected amount (65\%) being men and the remaining 35\% being women. the study found that from both of these groups, 55\% of the total play league for 2 -3 hours everyday. The NHS  \cite{NHS2021} defines obsession as "Unwanted or unpleasent thought, image or urge that repeatedly enters your mind, causing feelings of anxiety, digust or unease," which is in the context of Obessive Compulsive Disorder (OCD), though can be applied to gaming and other external areas.
\subsection{Impacts of Game Obsession}
The impacts of game addiciton, which could be applied to game obsession, involve forgoing basic needs in favour of playing the game in reclusion \cite{NHSHamp24}. 
\subsection{Causes of Game Obsession}

\section{Methodology}
\subsection{Ethics and safety}
\subsection {Hypothesis}
\subsection{Questions}
\subsection {Stats testing}


\section{Results}

\section {Conclusion}

\section*{References}

\bibliographystyle{plain}
\bibliography{Dissertationbib.bib}


\section {Addendum}

\end{document}
