\documentclass[conference]{IEEEtran}
\IEEEoverridecommandlockouts
% The preceding line is only needed to identify funding in the first footnote. If that is unneeded, please comment it out.
%Template version as of 6/27/2024

\usepackage{cite}
\usepackage{amsmath,amssymb,amsfonts}
\usepackage{algorithmic}
\usepackage{hyperref}
\usepackage{graphicx}
\usepackage{textcomp}
\usepackage{xcolor}
\def\BibTeX{{\rm B\kern-.05em{\sc i\kern-.025em b}\kern-.08em
    T\kern-.1667em\lower.7ex\hbox{E}\kern-.125emX}}
\begin{document}

\title{Game Obsession and Player engagement focused on mobile gaming enviroments*\\
{\footnotesize \textsuperscript{*}Note: Sub-titles are not captured for https://ieeexplore.ieee.org  and
should not be used}
%\thanks{Identify applicable funding agency here. If none, delete this.}
}

\author{\IEEEauthorblockN{1\textsuperscript{st} Marshall Sharp}
\IEEEauthorblockA{\textit{Games Academy} \\
\textit{Falmouth University}\\
Falmouth, United Kindgom \\
MS279226@falmouth.ac.uk}
}

\maketitle



\begin{abstract}

\end{abstract}

\begin{IEEEkeywords}
engagement, obsession, mobile games
\end{IEEEkeywords}

\section{Introduction}
The nature of obsession, most notably addiction, is a difficult thing to define concisely, as we don't know for certain what classifies as an addiction in an age of technological advancements. Over the years, the definition for addiction has been made to accomodate something which, in theory, cannot be applicable to it's original definition. It is most prevelent in the games industry, with game addiction being classified as a medical condition. This, however, could be flawed, this paper will take a look at game obsession instead of addiction, and go over game addiciton and player enagement based on game obsession instead of addiction, and challenging the classification of such terms.\\
\section{Background}
Gaming addiction can be defined in differing ways. One paper \cite{yasir2021} defines gaming addiction with the uses and gratification theory and defines overall addiction with the medical paradigm, which has a larger emphasis on biological and psychological dependacies, which has been retrofitted to accompany the advancements of technology, and specifically in the mobile games industry. this paper also mentions that another study conducted came to the conclusion that it can be defined as the regular action of taking drastic actions in the game, such as buying in game goods or features in a freenium game enviroment \cite{XWang2021}. \\

\cite{Naaj2021} was conducted during the midst of the COVID-19 pandemic in early 2020 at Ajman University, where it was found that from a sample size of 317, Game addiction was associated with bad academic performance, with the \\

The NHS in hampshire defines gaming addiction as a variety of symptoms, with the main symptoms being reclusion from the wider world, poor hygine or lack of attention to basic needs (i.e washing, nourishments such as food or water, or sleep) and hyperfixation on gaming \cite{NHSHamp24}. This definion can be seen as disagreeable due to the medical paradigm being more on a biological level rather than technological. In addition, it fails to take into consideration to other conditions in terms of symptoms, such as ADHD, Autism, and other neurodivergancies.\\

The term "Gaming Obsession" could fit this role better. One paper looks into Game Loyalty and Purchase intention \cite{Ramli2022}, in which it was found that Players who are loyal to the games they play are contradictory to a differing source they cited\cite{Widodo2020}, which has a trait of compulsive buying, which mirrors points made in \cite {yasir2021}.  This study was based on 350 out of 680 people, with the majority of the selected amount (65\%) being men and the remaining 35\% being women. the study found that from both of these groups, 55\% of the total play league for 2 -3 hours everyday. The NHS  \cite{NHS2021} defines obsession as "Unwanted or unpleasent thought, image or urge that repeatedly enters your mind, causing feelings of anxiety, digust or unease," which is in the context of Obessive Compulsive Disorder (OCD), though can be applied to gaming and other external areas.\\

\section{Methodology}
The Experiment for this research will involve creating a game similar to a Idle Game, Similar to games like cookie clicker \cite{CookieSteam21}, Where the Player can log in every so often to play the game and it requires bare minimum interaction from the player. According to \cite{Hwang2024}, the way players enagage with it is questionable, with them finding five key elements that are favoured\\
\newline
For half an hour, They will play the game for that time, or until they request that it ends early. An interview will happen after, going over the player's experiance with the game. Time played and actions per minute (APM) will be recorded. APM will be calculated as :\\

\begin{math}APM = Actions(A)/Time(T)\end{math}\\

The Artefact for the experiment will be made in Unity Version 2023.2.20 \cite{Unity23} and stored in this repository:\\

\hypertarget{https://github.falmouth.ac.uk/MS279226/Dissertation-MS279226.git}{https://github.falmouth.ac.uk/MS279226/Dissertation-MS279226.git}\\

\subsection{Ethics and safety}
There are some things to account for with this research, to do with people and their safety. The experiment is in the hands of the participant, with them being required to read a information form and sign a consent form. They can request the experiment to end early, If they no longer consent to the experiment.  If they no longer consent, data will not be used when drawing the conclusion. Despite there being little physical risk, a risk assessment will still need to be carried out, with the removal of items that can be harmful for participants taking part in the research, or the addition of items that can mitigate harm/ provide ease of use with the artifact.\\

\subsection {Hypothesis}
There are two main hypothesis with this experiment, with a third one to account for the possibility of it happening:

H1: The Player will have a hyperfocus on the game, and/or fulfill most of the criteria of game addiction mentioned in \cite{NHSHamp24}, though after being told it's over or requesting, they do not see a reason to keep going.

H2: The Player will not fulfill any criteria, and so there is no change plausible.

H3: The Player will fulfill the criteria of H1, but will continue with the experiment, even after it's over either through the duration being up or from requesting it to stop. This will fulfill all of the criteria in \cite{NHSHamp24}, and be classed as addiciton. \\
\subsection{Questions}
There are three main questions with this research:

Q1: How Long does it take for someone to consider themselves "Obessed" with the game?

Q2: Could/Should "Online Gaming Addiction" be renamed to "Online Gaming Obsession"?

Q3: What truly defines addiction? could addiction and obsession be one in the same?\\
\subsection {Stats testing}

\subsection {Restraints}
One restraint with this research is sample size. Due to my sample being students within the university. This limits the findings heavily, with there not being to get the sample size that will give a more noteworthy effect. In addition, the sample size will be limited even more due to the time period for the experiment only being three weeks. This is a similar pitfall to \cite{Naaj2021}.\\

%\section{Results}

%\section {Conclusion}

\section*{References}

\bibliographystyle{plain}
\bibliography{Dissertationbib.bib}


\section {Addendum}

\end{document}
